Upper limb musculoskeletal disorders (\textsc{ulmd}s) are the most prevalent occupational health problems affecting manual workers, with an incidence rate of about 32\% and 10 days away from work in the United States of America~\cite{Bureau_of_Labor_Statistics2015-jt,Luime2004-wt, Urwin1998-op}.
They result in increased production costs, lost work time, disability and may affect the worker’s quality of life~\cite{Eva1992-qg, Kuijpers2004-ue, Palmer2012-yc, Pope2001-ph}.
\textsc{Ulmd}s are common, and have many causes.
Combinations of poor technique~\cite{Miranda2008-gq}, forceful work, heavy lifting~\cite{Beach2012-yn}, repetitive~\cite{Harkness2003-qz} and overhead work~\cite{Leclerc2004-pd} are well-established physical risk factors.
A number of individual risk factors for \textsc{ulmd}s have been established as well.
Aside obesity~\cite{Luime2004-wt}, age~\cite{Boocock2015-ce} and coexisting medical conditions~\cite{Walker-Bone2005-dx}, being a female worker is associated with a higher prevalence of \textsc{ulmd}s~\cite{Hooftman2009-bk, Miranda2008-gq, Treaster2004-rr, Wahlstedt2010-wk}.
The biological characteristics that may explain the difference in the prevalence of upper limb injury between women and men are partially documented and are mainly related to anthropometry, muscle composition and strength differences~\cite{Cote2012-hn}.
Yet, the consequence of these biological differences on the movement's biomechanics remains unclear.

Much of the available evidence of the sex effect on occupational biomechanics comes from studies that either use systematic observation and interviews~\cite{Dahlberg2004-mw} or only consider the trunk and lower limbs~\cite{Lindbeck2001-fq, Plamondon2014-xe, Plamondon2017-qa}.
In our previous works, we investigated the sex-related difference in terms of kinematics~\cite{Martinez2019-mm} and electromyography~\cite{Bouffard2019-fd} during a dynamic manual handling task involving the upper limb.
The first study used a new kinematics indicator--the joint contribution--to show sex-related differences in lifting technique, which is a known risk factor of \textsc{ulmd}~\cite{Kilbom1998-la}.
We found that women used their elbows and wrists more to lift a 12~kg box between hip and eye levels compared to men.
Women’s handling technique seems to be altered when lifting a 6~kg box since they relied mostly on their glenohumeral joint.
These differences occurred when the box was above shoulder level and regardless of the mass lifted by men.
Using the same task, electromyography (\textsc{emg}) and muscle focus as an indicator of muscle coactivation, the second study showed that women generated a higher relative \textsc{emg} amplitude when lifting a 6 or 12~kg box compared to men.
Sex, however, did not seem to affect muscle coactivation.
While these findings highlight a sex specific lifting technique and muscle activation patterns, they do not draw a complete picture of the sex-related differences during a manual handling task involving the upper limb.

To better understand the link between the biomechanical indicators previously developed and the higher prevalence of \textsc{ulmd}s in women, quantitative assessments of musculoskeletal loads should be considered~\cite{Garg2009-wl}.
Several musculoskeletal models have been developed to estimate shoulder loading in ergonomic settings, with various degrees of complexity~\cite{Dickerson2007-qj, Pontonnier2014-vx}.
Although they are based on assumptions about the dynamics of human movement, they avoid in vivo measurement of shoulder joint loads that would be invasive and difficult to achieve in work settings.
Ergonomic models based on series of static postures, such as the 3D Static Strength Prediction Program, will underestimate internal forces~\cite{Garg2009-wl}, since they do not consider velocities and accelerations during dynamic tasks.
The development of the OpenSim musculoskeletal modeling platform~\cite{Delp2007-ol} has democratized the use of dynamic musculoskeletal models that are sufficiently detailed and usable to be relevant in occupational biomechanics~\cite{Kim2017-zr, Mortensen2018-jr}.

The current study aims to use an OpenSim musculoskeletal model of the upper limb to describe sex differences in musculoskeletal biomechanics during a dynamic and realistic lifting task.
We believe that the biological, kinematic and electromyographic differences previously described would lead to higher musculoskeletal loads in women compared to men.
The purpose of this investigation is twofold.
First, it would put into perspective the conclusions drawn from the kinematic and electromyographic indicators, which will help us understand the sex specificity of the lifting movement.
Second, this study could provide recommendations for effective technique that reduce exposure to \textsc{ulmd}s, as it was done for the back and the lower limbs~\cite{Plamondon2014-xe, Plamondon2017-qa}.